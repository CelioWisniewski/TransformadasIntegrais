%\documentclass[Main.tex]{subfiles}
%\begin{document}
\chapter{Representações da série de Fourier e diagramas de espectro}

No capítulo anterior, vimos que uma função periódica pode ser representa como uma série trigonométrica. No entanto, sobretudo em aplicações em Física e Engenharia, a série de Fourier é apresentada em outras formas, a forma harmônica (ou amplitude-fase) e a forma exponencial. Neste capítulo veremos como construir estas representações e introduziremos o conceito de diagramas de espectro de uma função periódica.

\section{Forma harmônica}
A forma harmônica, também chamada de forma amplitude-fase, da série de Fourier de uma função $f(t)$ é dada conforme a seguir:
$$
f(t)=A_0+\sum_{n=1}^\infty A_n\cos(w_n t-\theta_n),
$$
onde $w_n=\frac{2\pi n}{T}$, $A_n$ são constantes não negativas chamadas de amplitude e $\theta_n$ são ângulos de fase. Para relacionar esta representação com a forma trigonométrica, usamos a identidade trigonométrica
$$
\cos(a-b)=\cos(a)\cos(b)+\sen(a)\sen(b),
$$
com $a=w_n t$ e $b=\theta_n$. Assim temos:
\begin{eqnarray*}
f(t)&=&A_0+\sum_{n=1}^\infty A_n\cos(w_n t-\theta_n)\\
&=&A_0+\sum_{n=1}^\infty A_n\left[\cos(w_n t)\cos(\theta_n)+\sen(w_n t)\sen(\theta_n)\right]\\
&=&\underbrace{A_0}_{a_0/2}+\sum_{n=1}^\infty[ \underbrace{A_n\cos(\theta_n)}_{a_n} \cos(w_n t)+\underbrace{A_n\sen(\theta_n)}_{b_n}\sen(w_n t)]
\end{eqnarray*}
Comparando os termos da representação trigonométrica, temos que:
\begin{eqnarray*}
\frac{a_0}{2}&=&A_0\\
a_n&=&A_n\cos(\theta_n)\\
b_n&=&A_n\sen(\theta_n)\\
\end{eqnarray*}
Observe que
$$
a_n^2+b_n^2=A_n^2
$$
e, como $A_n\geq 0$ por hipótese, temos que 
$$
A_n=\sqrt{a_n^2+b_n^2}.
$$
Também temos
\begin{eqnarray*}
\cos(\theta_n)&=&\frac{a_n}{\sqrt{a_n^2+b_n^2}}\\
\sen(\theta_n)&=&\frac{b_n}{\sqrt{a_n^2+b_n^2}}
\end{eqnarray*}

Observe que sempre é possível converter uma forma na outra e os ângulos de fase estão unicamente definidos em cada volta do ciclo trigonométrico.

\begin{ex}Considere um função periódica ($T=4$) dada pelo gráfico
\begin{figure}[!ht]
\begin{center}
\psset{xunit =1cm,yunit=2cm, linewidth=1\pslinewidth}
 \begin{pspicture}(-1.3,-1.3)(4.3,1.3)
 \psaxes{->}(0,0)(-1.1,-1.2)(6.2,1.2)
\psplot[plotstyle=curve,linewidth=2\pslinewidth,linecolor=blue]{-1}{0}{0}
\psplot[plotstyle=curve,linewidth=2\pslinewidth,linecolor=blue]{0}{1}{1}
\psplot[plotstyle=curve,linewidth=2\pslinewidth,linecolor=blue]{1}{4}{0}
\psplot[plotstyle=curve,linewidth=2\pslinewidth,linecolor=blue]{4}{5}{1}
\psplot[plotstyle=curve,linewidth=2\pslinewidth,linecolor=blue]{5}{6}{0}
\rput(.3,1.3){$y=f(t)$}
\rput(6.1,.1){$t$}
\end{pspicture}
\end{center}
\end{figure}

Os coeficientes de Fourier são dados por 
\begin{eqnarray*}
\frac{a_0}{2}&=&\frac{1}{4}\int_0^4f(t)dt=\frac{1}{4}\int_0^1 1 dt=\frac{1}{4}\\[10pt]
a_n&=&\frac{2}{4}\int_0^4f(t)\cos(w_n t)dt=\frac{1}{2}\int_0^1\cos\left(\frac{\pi n}{2} t\right)dt=\frac{1}{\pi n}\left[\sen\left(\frac{\pi n}{2} t\right)\right]_0^1=\left\{\begin{array}{ll} 0,&n\ \hbox{par}\\[8pt]\frac{1}{\pi n}(-1)^{\frac{n-1}{2}}&n\ \hbox{ímpar}  \end{array}\right.\\[10pt]
b_n&=&\frac{2}{4}\int_0^4f(t)\sen(w_n t)dt=\frac{1}{2}\int_0^1\sen\left(\frac{\pi n}{2} t\right)dt=\frac{1}{\pi n}\left[-\cos\left(\frac{\pi n}{2} t\right)\right]_0^1=\left\{\begin{array}{ll} \frac{1}{\pi n},&n\ \hbox{ímpar}\\[8pt] \frac{1}{\pi n}\left(1-(-1)^{\frac{n}{2}}\right)&n\ \hbox{par}  \end{array}\right.\\
\end{eqnarray*}


\begin{table}[ht] 
\begin{center}
   \begin{tabular}{|c|c|c|c|c|}
   \hline
   $n$ & $a_n$ & $b_n$ & $A_n$&$\theta_n  $ \\
   \hline
   &&&&\\
   0& $\frac{1}{2}$ & 0 & $\frac{1}{4}$ & \\
   &&&&\\
   \hline
   &&&&\\
   1&$\frac{1}{\pi}$ & $\frac{1}{\pi}$ & $\frac{\sqrt{2}}{\pi}$ & $\frac{\pi}{4}$ \\
		&&&&\\
		\hline
		 &&&&\\
   2&$0$&$\frac{1}{\pi}$ & $\frac{1}{\pi}$ & $\frac{\pi}{2}$  \\
		&&&&\\
		\hline
		 &&&&\\
   3&$-\frac{1}{3\pi}$&$\frac{1}{3\pi}$& $\frac{\sqrt{2}}{3\pi}$ & $\frac{3\pi}{4}$\\
		&&&&\\
		\hline
		 &&&&\\
   4&$0$&$0$&0&\\
		&&&&\\
		\hline
		&&&&\\
		5&$\frac{1}{5\pi}$&$\frac{1}{5\pi}$&$\frac{\sqrt{2}}{5\pi}$&$\frac{\pi}{4}$ \\
		&&&&\\
		\hline
 \end{tabular}
 \caption{\label{tab_exponential_form}}
   \end{center}
\end{table}


Para escrever a forma harmônica da série de Fourier da função $f(t)$ calculamos as amplitudes $A_n$ e as fases $\theta_n$. Para $n=0$, temos $a_0=\frac{1}{2}$ e, portanto, $A_0=\frac{a_0}{2}=\frac{1}{4}$. Para $n=1$, temos $a_1=b_1=\frac{1}{\pi}$ e, consequentemente, $A_1=\sqrt{\frac{1}{\pi^2}+\frac{1}{\pi^2}}=\frac{\sqrt{2}}{\pi}$ e $\theta_1=\frac{\pi}{4}$. Os cálculos foram repetidos de forma análoga para $n=2,\ 3,\ 4$ e $5$ e apresentados na tabela \ref{tab_exponential_form}. Portanto,
$$
f(t)=\frac{1}{4}+\frac{1}{\pi}\left[\sqrt{2}\cos\left(\frac{\pi }{2} t-\frac{\pi}{4}\right)+\cos\left(\pi  t-\frac{\pi}{2}\right)+\frac{\sqrt{2}}{3}\cos\left(\frac{3 \pi }{2} t-\frac{3\pi}{4}\right)+\frac{\sqrt{2}}{5}\cos\left(\frac{5 \pi }{2} t-\frac{\pi }{4}\right)+\cdots   \right]
$$

\end{ex}

\section{Forma exponencial}
A forma exponencial de uma série de Fourier é obtida quando se substiuem as funções trigonométricas $\sen(w_nt)$ e $\cos(w_nt)$ por suas representações em termos de exponenciais complexos, isto é
$$\cos(w_nt)=\frac{e^{iw_nt}+ e^{-iw_nt}}{2}~~~\hbox{e}~~~\sen(w_nt)=\frac{e^{iw_nt}- e^{-iw_nt}}{2i}$$

\begin{eqnarray*}
f(t)&=&\frac{a_0}{2}+\sum_{n=1}^\infty a_n\cos(w_n t)+\sum_{n=1}^\infty b_n\sen(w_n t)\\
&=&\frac{a_0}{2}+\sum_{n=1}^\infty a_n\left(\frac{e^{iw_nt}+ e^{-iw_nt}}{2}\right)+\sum_{n=1}^\infty b_n\left(\frac{e^{iw_nt}- e^{-iw_nt}}{2i}\right)\\
\end{eqnarray*}
Reagrupando os termos e usando o fato que $\frac{1}{i}=-i$, temos:
\begin{eqnarray}\label{form_exp_1}
f(t)=\frac{a_0}{2}+\sum_{n=1}^\infty \frac{a_n-ib_n}{2}e^{iw_nt}+\sum_{n=1}^\infty \frac{a_n+ib_n}{2}e^{-iw_nt}
\end{eqnarray}
Agora observamos que as definições \ref{coef} dadas por  
\begin{eqnarray*}
   a_0&=& \frac{2}{T}\int_0^T f(t)dt = \frac{2}{T}\int_{-T/2}^{T/2} f(t)dt\\
   a_n&=& \frac{2}{T}\int_0^T f(t)\cos(w_n t)dt = \frac{2}{T}\int_{-T/2}^{T/2} f(t)\cos(w_nt)dt\\
   b_n&=& \frac{2}{T}\int_0^T f(t)\sen(w_n t)dt = \frac{2}{T}\int_{-T/2}^{T/2} f(t)\sen(w_nt)dt
  \end{eqnarray*}
Embora estas expressões estejam definadas apenas para $n>0$, elas fazem sentidos para qualquer $n$ inteiro. Neste caso, valem as seguintes identidades:
$$a_{-n}=a_n,~~b_{-n}=-b_{n}~~b_0=0.$$
onde se usou que $w_{-n}=\frac{2\pi (-n)}{T}=-\frac{2\pi n}{T}=-w_n$ e a paridade das funções cosseno e seno.

Estendendo estas definições para qualquer inteiro, introduzimos os coeficientes $C_n$ dados por:
\begin{equation}\label{def_cn}
C_n = \frac{a_n - ib_n}{2}
\end{equation}
Observe que estes coeficientes estão definidos para para número inteiro $n$, assim temos:
$$
C_0 = \frac{a_0 - ib_0}{2}=\frac{a_0}{2}
$$
e
$$
C_{-n} = \frac{a_{-n} - ib_{-n}}{2}=\frac{a_{n} + ib_{n}}{2}
$$
Substituindo estas expressões para $C_0$, $C_{n}$ e $C_{-n}$ em (\ref{form_exp_1}), obtemos:
\begin{eqnarray*}
f(t)=C_0+\sum_{n=1}^\infty C_n e^{iw_nt}+\sum_{n=1}^\infty C_{-n}e^{-iw_nt}
\end{eqnarray*}
Escrevemos agora esta última expressão em um único somatório:
\begin{eqnarray}\label{forma_exp}
f(t)=\sum_{n=-\infty}^\infty C_n e^{iw_nt}
\end{eqnarray}
onde se usou que $w_{-n}=\frac{2\pi (-n)}{T}=-\frac{2\pi n}{T}=-w_n$

Observamos também que os coeficientes $C_n$ podem ser escritos das seguinte forma mais enxuta:
\begin{eqnarray*}
C_n &=& \frac{a_n - ib_n}{2}\\
&=& \frac{1}{T}\int_0^Tf(t)\left[\cos(w_n t)-i\sen(w_n t)\right]dt\\
&=&\frac{1}{T}\int_0^Tf(t)e^{-iw_nt}dt =\frac{1}{T}\int_{-T/2}^{T/2}f(t)e^{-iw_nt}dt 
\end{eqnarray*}

\begin{ex}{\label{ex_exp_1}} A função $f(t)$ dada no exemplo \ref{ex_triangular} pode ser escrita na forma exponencial com os seguintes coeficientes:
$$C_0=\frac{a_0}{2}=\frac{1}{2}$$
$$C_n=\frac{a_n-ib_n}{2}=\frac{2\frac{(-1)^n-1}{\pi^2n^2}+0}{2}=\frac{(-1)^n-1}{\pi^2n^2},~~n\neq 0$$
 
\end{ex}

\begin{ex}{\label{ex_exp_2}} A função $g(t)$,
$$
g(t)=\frac{4}{\pi}\left(\sen(\pi t)+\frac{1}{3}\sen(3\pi t)+\frac{1}{5}\sen(5\pi t)+\cdots\right),
$$
calculada no exemplo \ref{ex_quadrada} pode ser escrita na forma exponencial com os seguintes coeficientes:
$$C_0=\frac{a_0}{2}=0$$
e
$$C_n=\frac{a_n-ib_n}{2}=\frac{0-i2\frac{1-(-1)^n}{\pi n}}{2}=i\frac{(-1)^n-1}{\pi n},~~n\neq 0.$$
Logo,
$$
g(t)=\cdots+\frac{2i}{5\pi}e^{-5i\pi t}+\frac{2i}{3\pi}e^{-3i\pi t}+\frac{2i}{\pi}e^{-i\pi t}-\frac{2i}{\pi}e^{i\pi t}-\frac{2i}{3\pi}e^{3i\pi t}-\frac{2i}{5\pi}e^{5i\pi t}-\cdots,
$$

\end{ex}

\section{Diagramas de espectro}
Diagramas espectro são representações gráficas dos coeficientes de Fourier $C_n$ associados a uma função periódica $f(t)$. Como os coeficientes $C_n$ são números complexos, é comum representá-los na forma de módulo e fase, isto é:
$$C_n = |C_n|e^{i\phi_n}.$$
O ângulo de fase assim definido coincide com o conceito de argumento do número $C_n$.

\begin{ex} A função 
$$f(t)=-1+2\cos(t)+4\sen(2t)$$
é periódica com periodo fundamental $2\pi$ e pode ser escrita na forma exponencial da seguinte forma:
\begin{eqnarray*}
f(t)&=&-1+2\left(\frac{e^{it}+e^{-it}}{2}\right)+4\left(\frac{e^{2it}-e^{-2it}}{2i}\right)\\
&=&2i e^{-2it} + e^{-it}-1+e^{it}- 2ie^{2it}
\end{eqnarray*}
Assim, identificamos cinco coeficientes não nulos:
\begin{equation*}
\begin{array}{lclcll}
 C_{-2}&=&2i=2e^{\frac{i\pi}{2}} &\Longrightarrow& |C_{-2}|=2, ~~ &\phi_{-2}=\frac{\pi}{2}\\
 C_{-1}&=&1 &\Longrightarrow& |C_{-1}|=1, ~~ &\phi_{-1}=0\\
 C_{0}&=&-1=1e^{\pi} &\Longrightarrow& |C_{0}|=1, ~~ &\phi_0=\pi\\
 C_{1}&=&1 &\Longrightarrow& |C_{1}|=1, ~~ &\phi_1=0\\
 C_{2}&=&-2i=2e^{\frac{-i\pi}{2}} &\Longrightarrow& |C_{2}|=2, ~~ &\phi_2=-\frac{\pi}{2}
\end{array}
 \end{equation*}

Os digramas de espectro de amplitude e fase são dados a seguir:
\begin{figure}[!ht]
 
  \begin{pspicture}(-6,-1.5) (6,2.6)
 \psset{xunit=1,yunit=1}
  \psaxes{->}(0,0)(-3.5,-.5)(3.5,2.5)

  \psline[linecolor=blue,linewidth=2pt]{-}(0,0)(0,1)
  
  \psline[linecolor=blue,linewidth=2pt]{-}(1,0)(1,1)
  \psline[linecolor=blue,linewidth=2pt]{-}(-1,0)(-1,1)

   \psline[linecolor=blue,linewidth=2pt]{-}(2,0)(2,2)
  \psline[linecolor=blue,linewidth=2pt]{-}(-2,0)(-2,2)

  
  \rput(.5,2.4){$|C_n|$}

  \rput(3.5,-.2){$w_n$}
\end{pspicture}

  \begin{pspicture}(-6,-2.6) (6,2.6)
 \psset{xunit=1,yunit=1}
  \psaxes[labels=x]{->}(0,0)(-3.5,-2.4)(3.5,2.4)

  \psline[linecolor=blue,linewidth=2pt]{-}(0,0)(0,2)
  
  %\psline[linecolor=blue,linewidth=2pt]{-}(1,0)(1,1)
  %\psline[linecolor=blue,linewidth=2pt]{-}(-1,0)(-1,1)

   \psline[linecolor=blue,linewidth=2pt]{-}(2,0)(2,-1)
  \psline[linecolor=blue,linewidth=2pt]{-}(-2,0)(-2,1)

  
  \rput(.5,2.4){$\phi_n$}

  \rput(3.5,-.2){$w_n$}
  \rput(-.3,1.0){$\frac{\pi}{2}$}
  \rput(-.5,-1.0){$-\frac{\pi}{2}$}
  \rput(-.3,2.0){$\pi$}
  \rput(-.5,-2.0){$-\pi$}
  \end{pspicture}
\end{figure}
\end{ex}
\begin{ex} As primeiras raias do digrama de espectro da função do exemplo \ref{ex_exp_2},
$$
g(t)=\cdots+\frac{2i}{5\pi}e^{-5i\pi t}+\frac{2i}{3\pi}e^{-3i\pi t}+\frac{2i}{\pi}e^{-i\pi t}-\frac{2i}{\pi}e^{i\pi t}-\frac{2i}{3\pi}e^{3i\pi t}-\frac{2i}{5\pi}e^{5i\pi t}-\cdots,
$$
são dados na figura a seguir

\begin{figure}[!ht] 
  \begin{pspicture}(-6,-1.5) (6,3.0)
 \psset{xunit=1,yunit=2}
  \psaxes[labels=none]{->}(0,0)(-5.5,-0.5)(5.5,1.3)

  
  
  \psline[linecolor=blue,linewidth=2pt]{-}(1,0)(1,1)
  \psline[linecolor=blue,linewidth=2pt]{-}(-1,0)(-1,1)

   \psline[linecolor=blue,linewidth=2pt]{-}(3,0)(3,.33)
  \psline[linecolor=blue,linewidth=2pt]{-}(-3,0)(-3,.33)

   \psline[linecolor=blue,linewidth=2pt]{-}(5,0)(5,.2)
  \psline[linecolor=blue,linewidth=2pt]{-}(-5,0)(-5,.2)

   \rput(-5.0,-.2){$-5\pi$}
   \rput(-4.0,-.2){$-4\pi$}
   \rput(-3.0,-.2){$-3\pi$}
   \rput(-2.0,-.2){$-2\pi$}
   \rput(-1.0,-.2){$-\pi$}
   \rput(1.0,-.2){$\pi$}
   \rput(2.0,-.2){$2\pi$}
   \rput(3.0,-.2){$3\pi$}
   \rput(4.0,-.2){$4\pi$}
   \rput(5.0,-.2){$5\pi$}
   
  
    \rput(-.3,1.0){$\frac{2}{\pi}$}
  \rput(0,1.5){$|C_n|$}

  \rput(5.7,-.1){$w_n$}
\end{pspicture}

  \begin{pspicture}(-6,-3.0) (6,3.0)
 \psset{xunit=1,yunit=1}
  \psaxes[labels=none]{->}(0,0)(-5.5,-2.4)(5.5,2.4)
  
  \psline[linecolor=blue,linewidth=2pt]{-}(1,0)(1,-1)
  \psline[linecolor=blue,linewidth=2pt]{-}(-1,0)(-1,1)

   \psline[linecolor=blue,linewidth=2pt]{-}(3,0)(3,-1)
  \psline[linecolor=blue,linewidth=2pt]{-}(-3,0)(-3,1)

   \psline[linecolor=blue,linewidth=2pt]{-}(5,0)(5,-1)
  \psline[linecolor=blue,linewidth=2pt]{-}(-5,0)(-5,1)

     \rput(-5.0,-.4){$-5\pi$}
   \rput(-4.0,-.4){$-4\pi$}
   \rput(-3.0,-.4){$-3\pi$}
   \rput(-2.0,-.4){$-2\pi$}
   \rput(-1.0,-.4){$-\pi$}
   \rput(1.0,.4){$\pi$}
   \rput(2.0,.4){$2\pi$}
   \rput(3.0,.4){$3\pi$}
   \rput(4.0,.4){$4\pi$}
   \rput(5.0,.4){$5\pi$}

  \rput(0,2.8){$\phi_n$}

  \rput(5.7,-.2){$w_n$}
  \rput(-.3,1.0){$\frac{\pi}{2}$}
  \rput(-.5,-1.0){$-\frac{\pi}{2}$}
  \rput(-.3,2.0){$\pi$}
  \rput(-.5,-2.0){$-\pi$}
  \end{pspicture}
\end{figure}
\end{ex}

\section{Exercícios}

\begin{Exercise}
Esboce os diagramas de amplitude e fase do espectro das seguintes funções periódicas:
\begin{itemize}
\item[a)] $f(t)=\sen(t)$\
\item[b)] $f(t)=3\cos(\pi t)$
\item[c)] $f(t)=1+4\cos(\pi t)$
\item[d)] $f(t)=2\cos^2(2\pi t)$
\item[e)] $f(t)=8\sen^3(2\pi t)+2\cos(6\pi t)$
\item[f)] $f(t)=\sen(2\pi t)+\cos(3\pi t)$
\end{itemize}
Observação: Considere a fase $\phi$ no intervalo $-\pi< \phi\leq \pi$
\end{Exercise}



\begin{Answer}
 \begin{itemize}
  \item [a)]
 Observe que $$\sen(t)=\frac{1}{2i}\left(e^{it}-e^{-it}\right)= \frac{i}{2}e^{-it} - \frac{i}{2}e^{it}$$ e a frequência angular fundamental é $w_F=1$. Veja os diagramas de espectro na figura abaixo.
 
  \begin{pspicture}(-6,-1.5) (6,2.6)
 \psset{xunit=1,yunit=1}
  \psaxes{->}(0,0)(-3.5,-.5)(3.5,2.5)
  \psline[linecolor=blue,linewidth=2pt]{-}(1,0)(1,.5)
  \psline[linecolor=blue,linewidth=2pt]{-}(-1,0)(-1,.5)
  \rput(.5,2.4){$|C_n|$}
  \rput(3.5,-.2){$w_n$}
\end{pspicture}

  \begin{pspicture}(-6,-2.6) (6,2.6)
 \psset{xunit=1,yunit=1}
  \psaxes[labels=x]{->}(0,0)(-3.5,-2.4)(3.5,2.4)
  \psline[linecolor=blue,linewidth=2pt]{-}(-1,0)(-1,1)
  \psline[linecolor=blue,linewidth=2pt]{-}(1,0)(1,-1)
  
  \rput(.5,2.4){$\phi_n$}

  \rput(3.5,-.2){$w_n$}
  \rput(-.3,2.0){$\pi$}
  \rput(-.5,-2.0){$-\pi$}
  \end{pspicture}


  \item [b)]
 Observe que $$3\cos(\pi t)=\frac{3}{2}\left(e^{i\pi t}+e^{-i\pi t}\right)= \frac{3}{2}e^{-i\pi t} + \frac{3}{2}e^{i\pi t}$$ e a frequência angular fundamental é $w_F=\pi$. Veja os diagramas de espectro na figura abaixo. 
  \begin{pspicture}(-6,-1.5) (6,2.6)
 \psset{xunit=1,yunit=1}
  \psaxes[labels=y]{->}(0,0)(-3.5,-.5)(3.5,2.5)
  \psline[linecolor=blue,linewidth=2pt]{-}(1,0)(1,1.5)
  \psline[linecolor=blue,linewidth=2pt]{-}(-1,0)(-1,1.5)
  \rput(.5,2.4){$|C_n|$}
  \rput(3.5,-.2){$w_n$}
  
  \rput(-2.2,-.4){$-2\pi$}
  \rput(-1.2,-.4){$-\pi$}
    \rput(1,-.4){$\pi$}
  \rput(2,-.4){$2\pi$}
  \end{pspicture}

  \begin{pspicture}(-6,-2.6) (6,2.6)
 \psset{xunit=1,yunit=1}
  \psaxes[labels=none]{->}(0,0)(-3.5,-2.4)(3.5,2.4)
  \psset{linecolor=blue}

  \qdisk(1,0){.1}
  \qdisk(-1,0){.1}
  \rput(.5,2.4){$\phi_n$}

  \rput(3.5,-.2){$w_n$}
  \rput(-.3,2.0){$\pi$}
  
  \rput(-.5,-2.0){$-\pi$}
  \rput(-2.2,-.4){$-2\pi$}
  \rput(-1.2,-.4){$-\pi$}
    \rput(1,-.4){$\pi$}
  \rput(2,-.4){$2\pi$}
  
  \end{pspicture}

\item [c)]
 Observe que $$1+4\cos(\pi t)=1+2\left(e^{i\pi t}+e^{-i\pi t}\right)= 1+2e^{-i\pi t} + 2e^{i\pi t}$$ e a frequência angular fundamental é $w_F=\pi$.  Veja os diagramas de espectro na figura abaixo. 

  \begin{pspicture}(-6,-1.5) (6,2.6)
 \psset{xunit=1,yunit=1}
  \psaxes[labels=y]{->}(0,0)(-3.5,-.5)(3.5,2.5)
  \psline[linecolor=blue,linewidth=2pt]{-}(1,0)(1,2)
  \psline[linecolor=blue,linewidth=2pt]{-}(0,0)(0,1)
	\psline[linecolor=blue,linewidth=2pt]{-}(-1,0)(-1,2)
  \rput(.5,2.4){$|C_n|$}
  \rput(3.5,-.2){$w_n$}
  
  \rput(-2.2,-.4){$-2\pi$}
  \rput(-1.2,-.4){$-\pi$}
    \rput(1,-.4){$\pi$}
  \rput(2,-.4){$2\pi$}

\end{pspicture}

  \begin{pspicture}(-6,-2.6) (6,2.6)
 \psset{xunit=1,yunit=1}
  \psaxes[labels=none]{->}(0,0)(-3.5,-2.4)(3.5,2.4)
  \psset{linecolor=blue}

  \qdisk(1,0){.1}
  \qdisk(0,0){.1}
	  \qdisk(-1,0){.1}
  \rput(.5,2.4){$\phi_n$}

  \rput(3.5,-.2){$w_n$}
  \rput(-.3,2.0){$\pi$}
  
  \rput(-.5,-2.0){$-\pi$}
  \rput(-2.2,-.4){$-2\pi$}
  \rput(-1.2,-.4){$-\pi$}
    \rput(1,-.4){$\pi$}
  \rput(2,-.4){$2\pi$}
  
  \end{pspicture}

\item [d)]
 Observe que $$2\cos^2(2\pi t)=2\left(\frac{e^{2i\pi t}+e^{-2i\pi t}}{2}\right)^2= \frac{e^{-4i\pi t} +2+ e^{4i\pi t}}{2}= \frac{1}{2}e^{-4i\pi t} +1+ \frac{1}{2}e^{4i\pi t}$$ e a frequência angular fundamental é $w_F=4\pi$.  Veja os diagramas de espectro na figura abaixo. 
 
  \begin{pspicture}(-6,-1.5) (6,1.6)
 \psset{xunit=1,yunit=1}
  \psaxes[labels=y]{->}(0,0)(-3.5,-.5)(3.5,1.5)
  \psline[linecolor=blue,linewidth=2pt]{-}(2,0)(2,0.5)
  \psline[linecolor=blue,linewidth=2pt]{-}(0,0)(0,1)
	\psline[linecolor=blue,linewidth=2pt]{-}(-2,0)(-2,0.5)
  \rput(.5,1.5){$|C_n|$}
  \rput(3.5,-.2){$w_n$}
  
  \rput(-2.2,-.4){$-4\pi$}
  \rput(-1.2,-.4){$-2\pi$}
    \rput(1,-.4){$2\pi$}
  \rput(2,-.4){$4\pi$}
  
\end{pspicture}

  \begin{pspicture}(-6,-2.6) (6,2.6)
 \psset{xunit=1,yunit=1}
  \psaxes[labels=none]{->}(0,0)(-3.5,-2.4)(3.5,2.4)
  \psset{linecolor=blue}

  \qdisk(2,0){.1}
  \qdisk(0,0){.1}
	  \qdisk(-2,0){.1}
  \rput(.5,2.4){$\phi_n$}

  \rput(3.5,-.2){$w_n$}
  \rput(-.3,2.0){$\pi$}
  \rput(-.5,-2.0){$-\pi$}
  
	\rput(-2.2,-.4){$-4\pi$}
  \rput(-1.2,-.4){$-2\pi$}
    \rput(1,-.4){$2\pi$}
  \rput(2,-.4){$4\pi$}
  
  \end{pspicture}

\item [e)]
 Observe que 
\begin{eqnarray*}
8\sen^3(2\pi t)+2\cos(6\pi t)&=&8\left(\frac{e^{2i\pi t}-e^{-2i\pi t}}{2i}\right)^3+2\left(\frac{e^{6i\pi t}+e^{-6i\pi t}}{2}\right)\\&=& (i+1)e^{6i\pi t}-3 i e^{2i\pi t}+3 i e^{-2i\pi t}+(1-i)e^{-6i\pi t}\\&=&\sqrt{2}e^{\frac{\pi}{4}i} e^{6i\pi t}+3e^{-\frac{\pi}{2}i} e^{2i\pi t}+3 e^{\frac{\pi}{2}i} e^{-2i\pi t}+\sqrt{2}e^{-\frac{\pi}{4}i}e^{-6i\pi t} 
\end{eqnarray*}
e a frequência angular fundamental é $w_F=2\pi$.  Veja os diagramas de espectro na figura abaixo.
 
  \begin{pspicture}(-6,-1.5) (6,3.7)
 \psset{xunit=1,yunit=1}
  \psaxes[labels=y]{->}(0,0)(-3.5,-.5)(3.5,3.5)
  \psline[linecolor=blue,linewidth=2pt]{-}(-3,0)(-3,1.4)
  \psline[linecolor=blue,linewidth=2pt]{-}(-1,0)(-1,3)
	\psline[linecolor=blue,linewidth=2pt]{-}(1,0)(1,3)
	\psline[linecolor=blue,linewidth=2pt]{-}(3,0)(3,1.4)
  \rput(.5,3.4){$|C_n|$}
  \rput(3.5,-.2){$w_n$}
  
	\rput(-3.2,-.4){$-6\pi$}
  \rput(-2.2,-.4){$-4\pi$}
  \rput(-1.2,-.4){$-2\pi$}
    \rput(1,-.4){$2\pi$}
  \rput(2,-.4){$4\pi$}
   \rput(3,-.4){$6\pi$}
	
\end{pspicture}

  \begin{pspicture}(-6,-2.6) (6,2.6)
 \psset{xunit=1,yunit=1}
  \psaxes[labels=none]{->}(0,0)(-3.5,-2.4)(3.5,2.4)
  \psset{linecolor=blue}


\psline[linecolor=blue,linewidth=2pt]{-}(-3,0)(-3,-0.5)
\psline[linecolor=blue,linewidth=2pt]{-}(-1,0)(-1,1.0)
\psline[linecolor=blue,linewidth=2pt]{-}(1,0)(1,-1.0)
	\psline[linecolor=blue,linewidth=2pt]{-}(3,0)(3,0.5)
	
  \rput(.5,2.4){$\phi_n$}

  \rput(3.5,-.2){$w_n$}
  \rput(-.3,2.0){$\pi$}
  \rput(-.5,-2.0){$-\pi$}
  
		\rput(-3.2,.4){$-6\pi$}
	\rput(-2.2,-.4){$-4\pi$}
  \rput(-1.2,-.4){$-2\pi$}
    \rput(1,.4){$2\pi$}
  \rput(2,-.4){$4\pi$}
  \rput(3,-.4){$6\pi$}
  \end{pspicture}


\item [f)]
 Observe que 
\begin{eqnarray*}
\sen(2\pi t)+\cos(3\pi t)&=&\left(\frac{e^{2i\pi t}-e^{-2i\pi t}}{2i}\right)+\left(\frac{e^{3i\pi t}+e^{-3i\pi t}}{2}\right)\\
&=& -\frac{i}{2}e^{2i\pi t}+\frac{i}{2}e^{-2i\pi t}+\frac{1}{2}e^{3i\pi t}+\frac{1}{2}e^{-3i\pi t}
\end{eqnarray*}
e a frequência angular fundamental é $w_F=\pi$ (ver exercício \ref{freq_fund} na página \pageref{freq_fund}).  Veja os diagramas de espectro na figura abaixo.

  \begin{pspicture}(-6,-1.5) (6,1.6)
 \psset{xunit=1,yunit=1}
  \psaxes[labels=y]{->}(0,0)(-3.5,-.5)(3.5,1.5)
  \psline[linecolor=blue,linewidth=2pt]{-}(-3,0)(-3,0.5)
  \psline[linecolor=blue,linewidth=2pt]{-}(-2,0)(-2,0.5)
	\psline[linecolor=blue,linewidth=2pt]{-}(2,0)(2,0.5)
	\psline[linecolor=blue,linewidth=2pt]{-}(3,0)(3,0.5)
  \rput(.5,1.4){$|C_n|$}
  \rput(3.5,-.2){$w_n$}
  
	\rput(-3.2,-.4){$-3\pi$}
  \rput(-2.2,-.4){$-2\pi$}
  \rput(-1.2,-.4){$-\pi$}
    \rput(1,-.4){$\pi$}
  \rput(2,-.4){$2\pi$}
   \rput(3,-.4){$3\pi$}
	
\end{pspicture}

  \begin{pspicture}(-6,-2.6) (6,2.6)
 \psset{xunit=1,yunit=1}
  \psaxes[labels=none]{->}(0,0)(-3.5,-2.4)(3.5,2.4)
  \psset{linecolor=blue}



\psline[linecolor=blue,linewidth=2pt]{-}(-2,0)(-2,1.0)
\psline[linecolor=blue,linewidth=2pt]{-}(2,0)(2,-1.0)

  \qdisk(-3,0){.1}
	  \qdisk(3,0){.1}
	
  \rput(.5,2.4){$\phi_n$}

  \rput(3.5,-.2){$w_n$}
  \rput(-.3,2.0){$\pi$}
  \rput(-.5,-2.0){$-\pi$}
  
		\rput(-3.2,.4){$-3\pi$}
	\rput(-2.2,-.4){$-2\pi$}
  \rput(-1.2,-.4){$-\pi$}
    \rput(1,.4){$\pi$}
  \rput(2,-.4){$2\pi$}
  \rput(3,-.4){$3\pi$}
  \end{pspicture}



\end{itemize}
\end{Answer}

\begin{Exercise}
Esboce os diagramas de amplitude e fase do espectro, indicando pelo menos as cinco primeiras raias positivas e negativas, das seguintes funções periódicas:
\begin{itemize}
\item[a)] $f(t)=\sum_{n=-\infty}^\infty \frac{e^{i \pi n t}}{n^2+1}$
\item[b)] $f(t)=\sum_{n=1}^\infty \frac{\sen(nt)}{n^2}$
\end{itemize}
\end{Exercise}

\begin{Answer} 
\begin{itemize}
\item [a)] Observe que $f(t)$ já está na forma exponencial e a frequência fundamental é $w_F=\pi$. Também temos:
$$
\begin{array}{|c|c|c|c|}
\hline
n&\omega_n&|C_n|&\phi_n\\
\hline
-5&-5\pi &\frac{1}{(-5)^2+1}=\frac{1}{26}&0\\
\hline
-4&-4\pi &\frac{1}{(-4)^2+1}=\frac{1}{17}&0\\
\hline
-3&-3\pi &\frac{1}{(-3)^2+1}=\frac{1}{10}&0\\
\hline
-2&-2\pi &\frac{1}{(-2)^2+1}=\frac{1}{5}&0\\
\hline
-1&-\pi &\frac{1}{(-1)^2+1}=\frac{1}{2}&0\\
\hline
0&0 &\frac{1}{(0)^2+1}=1&0\\
\hline
1&1\pi &\frac{1}{1^2+1}=\frac{1}{2}&0\\
\hline
2&2\pi &\frac{1}{2^2+1}=\frac{1}{5}&0\\
\hline
3&3\pi &\frac{1}{3^2+1}=\frac{1}{10}&0\\
\hline
4&4\pi &\frac{1}{4^2+1}=\frac{1}{17}&0\\
\hline
5&5\pi &\frac{1}{5^2+1}=\frac{1}{26}&0\\
\hline
\end{array}
$$
Veja o diagrama de amplitude na figura abaixo.

  \begin{pspicture}(-6,-0.5) (6,3.6)
 \psset{xunit=1,yunit=3}
  \psaxes[labels=y]{->}(0,0)(-5.5,-.2)(5.7,1.2)
  \psline[linecolor=blue,linewidth=2pt]{-}(-5,0)(-5,0.038)
  \psline[linecolor=blue,linewidth=2pt]{-}(-4,0)(-4,0.059)
	\psline[linecolor=blue,linewidth=2pt]{-}(-3,0)(-3,0.1)
	\psline[linecolor=blue,linewidth=2pt]{-}(-2,0)(-2,0.2)
	\psline[linecolor=blue,linewidth=2pt]{-}(-1,0)(-1,0.5)
	\psline[linecolor=blue,linewidth=2pt]{-}(0,0)(0,1.0)
  \psline[linecolor=blue,linewidth=2pt]{-}(5,0)(5,0.038)
  \psline[linecolor=blue,linewidth=2pt]{-}(4,0)(4,0.059)
	\psline[linecolor=blue,linewidth=2pt]{-}(3,0)(3,0.1)
	\psline[linecolor=blue,linewidth=2pt]{-}(2,0)(2,0.2)
	\psline[linecolor=blue,linewidth=2pt]{-}(1,0)(1,0.5)


  \rput(.5,1.1){$|C_n|$}
  \rput(5.5,-.1){$w_n$}
  
	\rput(-5.2,-.1){$-5\pi$}
  \rput(-4.2,-.1){$-4\pi$}
	\rput(-3.2,-.1){$-3\pi$}
  \rput(-2.2,-.1){$-2\pi$}
  \rput(-1.2,-.1){$-\pi$}
    \rput(1,-.1){$\pi$}
  \rput(2,-.1){$2\pi$}
   \rput(3,-.1){$3\pi$}
	\rput(4,-.1){$4\pi$}
   \rput(5,-.1){$5\pi$}
	
\end{pspicture}





\item [b)]
 Começamos escrevendo a função $f(t)=\sum_{n=1}^\infty \frac{\sen(nt)}{n^2}$ na forma exponencial:
\begin{eqnarray*}
\sum_{n=1}^\infty \frac{\sen(nt)}{n^2}&=&\sum_{n=1}^\infty \frac{1}{n^2}\left(\frac{e^{int}-e^{-int}}{2i}\right)\\
&=&\sum_{n=1}^\infty \frac{1}{2i n^2} e^{int}+\sum_{n=1}^\infty\left( -\frac{1}{2i n^2}e^{-int}\right)\\
&=&\sum_{n=1}^\infty \left(-\frac{i}{2 n^2} e^{int}\right)+\sum_{n=-1}^{-\infty}\frac{i}{2 n^2}e^{int}.
\end{eqnarray*}
A frequência angular fundamental é $w_F=1$ e as amplitudes e fases são dados na tabela abaixo.
$$
\begin{array}{|c|c|c|}
\hline
\omega_n=n&|C_n|&\phi_n\\
\hline
-5&\frac{1}{50}&\frac{\pi}{2}\\
\hline
-4 &\frac{1}{32}&\frac{\pi}{2}\\
\hline
-3 &\frac{1}{18}&\frac{\pi}{2}\\
\hline
-2 &\frac{1}{8}&\frac{\pi}{2}\\
\hline
-1 &\frac{1}{2}&\frac{\pi}{2}\\
\hline
0 &0&-\\
\hline
1 &\frac{1}{2}&-\frac{\pi}{2}\\
\hline
2 &\frac{1}{8}&-\frac{\pi}{2}\\
\hline
3 &\frac{1}{18}&-\frac{\pi}{2}\\
\hline
4 &\frac{1}{32}&-\frac{\pi}{2}\\
\hline
5 &\frac{1}{50}&-\frac{\pi}{2}\\
\hline
\end{array}
$$
Veja os diagramas de espectro na figura abaixo.

  \begin{pspicture}(-6,-1.5) (6,2.3)
 \psset{xunit=1,yunit=3}
  \psaxes[labels=x]{->}(0,0)(-5.5,-.2)(5.5,0.7)
	
  \psline[linecolor=blue,linewidth=2pt]{-}(-5,0)(-5,.02)
	\psline[linecolor=blue,linewidth=2pt]{-}(-4,0)(-4,0.03125)
  \psline[linecolor=blue,linewidth=2pt]{-}(-3,0)(-3,0.055556)
	\psline[linecolor=blue,linewidth=2pt]{-}(-2,0)(-2,0.125)
	\psline[linecolor=blue,linewidth=2pt]{-}(-1,0)(-1,0.5)
	\psset{linecolor=blue}
	\qdisk(0,0){.1}
	\psline[linecolor=blue,linewidth=2pt]{-}(5,0)(5,.02)
	\psline[linecolor=blue,linewidth=2pt]{-}(4,0)(4,0.03125)
  \psline[linecolor=blue,linewidth=2pt]{-}(3,0)(3,0.055556)
	\psline[linecolor=blue,linewidth=2pt]{-}(2,0)(2,0.125)
	\psline[linecolor=blue,linewidth=2pt]{-}(1,0)(1,0.5)
	
	
  \rput(.5,0.6){$|C_n|$}
  \rput(5.5,-.1){$w_n$}
  \rput(-0.3,0.5){$\frac{1}{2}$}
	
	
\end{pspicture}

  \begin{pspicture}(-6,-2.6) (6,2.6)
 \psset{xunit=1,yunit=1}
  \psaxes[labels=none]{->}(0,0)(-5.5,-2.4)(5.5,2.4)
  \psset{linecolor=blue}



\psline[linecolor=blue,linewidth=2pt]{-}(-5,0)(-5,1.0)
\psline[linecolor=blue,linewidth=2pt]{-}(-4,0)(-4,1.0)
\psline[linecolor=blue,linewidth=2pt]{-}(-3,0)(-3,1.0)
\psline[linecolor=blue,linewidth=2pt]{-}(-2,0)(-2,1.0)
\psline[linecolor=blue,linewidth=2pt]{-}(-1,0)(-1,1.0)
\psline[linecolor=blue,linewidth=2pt]{-}(1,0)(1,-1.0)
\psline[linecolor=blue,linewidth=2pt]{-}(2,0)(2,-1.0)
\psline[linecolor=blue,linewidth=2pt]{-}(3,0)(3,-1.0)
\psline[linecolor=blue,linewidth=2pt]{-}(4,0)(4,-1.0)
\psline[linecolor=blue,linewidth=2pt]{-}(5,0)(5,-1.0)


  \rput(.5,2.4){$\phi_n$}

  \rput(5.5,-.2){$w_n$}
	
		\rput(1.0,.4){$1$}
  \rput(2.0,.4){$2$}
	\rput(3.0,.4){$3$}
  \rput(4.0,.4){$4$}
  \rput(5.0,.4){$5$}

		\rput(-1.0,-.4){$-1$}
  \rput(-2.0,-.4){$-2$}
	\rput(-3.0,-.4){$-3$}
  \rput(-4.0,-.4){$-4$}
  \rput(-5.0,-.4){$-5$}

  \rput(-.3,2.0){$\pi$}
  \rput(-.5,-2.0){$-\pi$}
 
  \end{pspicture}



\end{itemize}
\end{Answer}

\begin{Exercise}Esboce os diagramas de espectro das séries de Fourier dos problemas \ref{Fourier_8} e \ref{Fourier_9} da página \pageref{Fourier_8}.
\end{Exercise}

\begin{Answer}
\begin{itemize}
\item problema \ref{Fourier_8}, a)
\begin{eqnarray*}
f(t)&=&\frac{2}{\pi}- \frac{4}{\pi}\sum_{n=1}^\infty \frac{\cos(2n\pi t)}{4n^2-1}\\
&=&\frac{2}{\pi}- \frac{4}{\pi}\sum_{n=1}^\infty \frac{1}{4n^2-1}\left(\frac{e^{2 n\pi it}+e^{-2n\pi it}}{2}\right)\\
&=&\frac{2}{\pi}- \sum_{n=1}^\infty \frac{2}{\pi(4n^2-1)}e^{2 n\pi it}- \sum_{n=-1}^{-\infty} \frac{2}{\pi(4n^2-1)}e^{2n\pi it}
  \end{eqnarray*}
Veja os diagramas de espectro na figura abaixo.
 
  \begin{pspicture}(-6,-1.5) (6,2.3)
 \psset{xunit=1,yunit=3}
  \psaxes[labels=none]{->}(0,0)(-3.5,-.2)(3.5,0.8)
	
  \psline[linecolor=blue,linewidth=2pt]{-}(-3,0)(-3,0.0182)
	\psline[linecolor=blue,linewidth=2pt]{-}(-2,0)(-2,0.042)
	\psline[linecolor=blue,linewidth=2pt]{-}(-1,0)(-1,0.2122)
	\psline[linecolor=blue,linewidth=2pt]{-}(0,0)(0,0.635)
	\psline[linecolor=blue,linewidth=2pt]{-}(1,0)(1,.2122)
	\psline[linecolor=blue,linewidth=2pt]{-}(2,0)(2,0.042)
  \psline[linecolor=blue,linewidth=2pt]{-}(3,0)(3,0.0182)
	
	
  \rput(.5,0.7){$|C_n|$}
  \rput(3.5,-.1){$w_n$}
  \rput(-0.3,0.6){$\frac{2}{\pi}$}
	
		\rput(1.0,-.1){$2\pi$}
  \rput(2.0,-.1){$4\pi$}
	\rput(3.0,-.1){$6\pi$}
  
		\rput(-1.0,-.1){$-2\pi$}
  \rput(-2.0,-.1){$-4\pi$}
	\rput(-3.0,-.1){$-6\pi$}
\end{pspicture}

  \begin{pspicture}(-6,-2.6) (6,2.6)
 \psset{xunit=1,yunit=1}
  \psaxes[labels=none]{->}(0,0)(-3.5,-2.4)(3.5,2.4)
  \psset{linecolor=blue}



\psline[linecolor=blue,linewidth=2pt]{-}(-3,0)(-3,2.0)
\psline[linecolor=blue,linewidth=2pt]{-}(-2,0)(-2,2.0)
\psline[linecolor=blue,linewidth=2pt]{-}(-1,0)(-1,2.0)
\psline[linecolor=blue,linewidth=2pt]{-}(1,0)(1,-2.0)
\psline[linecolor=blue,linewidth=2pt]{-}(2,0)(2,-2.0)
\psline[linecolor=blue,linewidth=2pt]{-}(3,0)(3,-2.0)
\qdisk(0,0){.1}

  \rput(.5,2.4){$\phi_n$}

  \rput(3.5,-.2){$w_n$}
	
		\rput(1.0,.4){$2\pi$}
  \rput(2.0,.4){$4\pi$}
	\rput(3.0,.4){$6\pi$}
  
		\rput(-1.0,-.4){$-2\pi$}
  \rput(-2.0,-.4){$-4\pi$}
	\rput(-3.0,-.4){$-6\pi$}

  \rput(-.3,2.0){$\pi$}
  \rput(-.5,-2.0){$-\pi$}
 
  \end{pspicture}


\item problema \ref{Fourier_9}
\begin{eqnarray*}
h(t)&=&\frac{2}{\pi}- \frac{4}{\pi}\sum_{n=1}^\infty (-1)^n\frac{\cos(2n\pi t)}{4n^2-1}\\
&=&\frac{2}{\pi}- \frac{4}{\pi}\sum_{n=1}^\infty \frac{(-1)^n}{4n^2-1}\left(\frac{e^{2 n\pi it}+e^{-2n\pi it}}{2}\right)\\
&=&\frac{2}{\pi}- \sum_{n=1}^\infty \frac{2(-1)^n}{\pi(4n^2-1)}e^{2 n\pi it}- \sum_{n=-1}^{-\infty} \frac{2(-1)^n}{\pi(4n^2-1)}e^{2n\pi it}
  \end{eqnarray*}
	Veja os diagramas de espectro na figura abaixo.

  \begin{pspicture}(-6,-1.5) (6,2.3)
 \psset{xunit=1,yunit=3}
  \psaxes[labels=none]{->}(0,0)(-3.5,-.2)(3.5,0.8)
	
  \psline[linecolor=blue,linewidth=2pt]{-}(-3,0)(-3,0.0182)
	\psline[linecolor=blue,linewidth=2pt]{-}(-2,0)(-2,0.042)
	\psline[linecolor=blue,linewidth=2pt]{-}(-1,0)(-1,0.2122)
	\psline[linecolor=blue,linewidth=2pt]{-}(0,0)(0,0.635)
	\psline[linecolor=blue,linewidth=2pt]{-}(1,0)(1,.2122)
	\psline[linecolor=blue,linewidth=2pt]{-}(2,0)(2,0.042)
  \psline[linecolor=blue,linewidth=2pt]{-}(3,0)(3,0.0182)
	
	
  \rput(.5,0.7){$|C_n|$}
  \rput(3.5,-.1){$w_n$}
  \rput(-0.3,0.6){$\frac{2}{\pi}$}
	
		\rput(1.0,-.1){$2\pi$}
  \rput(2.0,-.1){$4\pi$}
	\rput(3.0,-.1){$6\pi$}
  
		\rput(-1.0,-.1){$-2\pi$}
  \rput(-2.0,-.1){$-4\pi$}
	\rput(-3.0,-.1){$-6\pi$}
\end{pspicture}

  \begin{pspicture}(-6,-2.6) (6,2.6)
 \psset{xunit=1,yunit=1}
  \psaxes[labels=none]{->}(0,0)(-3.5,-2.4)(3.5,2.4)
  \psset{linecolor=blue}



\psline[linecolor=blue,linewidth=2pt]{-}(-2,0)(-2,2.0)
\qdisk(-3,0){.1}

\qdisk(-1,0){.1}
\qdisk(0,0){.1}

\qdisk(1,0){.1}
\psline[linecolor=blue,linewidth=2pt]{-}(2,0)(2,2.0)
\qdisk(3,0){.1}

  \rput(.5,2.4){$\phi_n$}

  \rput(3.5,-.2){$w_n$}
	
		\rput(1.0,-.4){$2\pi$}
  \rput(2.0,-.4){$4\pi$}
	\rput(3.0,-.4){$6\pi$}
  
		\rput(-1.0,-.4){$-2\pi$}
  \rput(-2.0,-.4){$-4\pi$}
	\rput(-3.0,-.4){$-6\pi$}

  \rput(-.3,2.0){$\pi$}
  \rput(-.5,-2.0){$-\pi$}
 
  \end{pspicture}

\end{itemize}
\end{Answer}

\begin{Exercise} Mostre que se $f(t)$ é uma função real, então $C_{-n}=\overline{C_n}$. Em especial, $|C_{-n}|=|C_n|$.
\end{Exercise}


\begin{Exercise} Mostre que se $f(t)$ é um deslocamento no tempo de $g(t)$, isto é, $f(t)=g(t-k)$, então os coeficiente de Fourier $C_n^f$ da função $f$ e $C_n^g$ da função $g$ são iguais em módulo e, portanto, possuem o mesmo diagrama de espectro de amplitude.
\end{Exercise}



%\end{document}