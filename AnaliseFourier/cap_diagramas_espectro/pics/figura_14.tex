
 \documentclass{standalone}
 \usepackage{pstricks-add}
 \usepackage{pstricks,pst-plot}
 \usepackage[dvips]{graphicx}
 \usepackage{pst-math}
 \usepackage{pst-plot}
 \usepackage{pst-circ}
 \usepackage[brazil]{babel}
 \usepackage[utf8]{inputenc}
 \usepackage[T1]{fontenc}
 \usepackage{amsmath}
 \usepackage{amssymb}
 \usepackage{amsthm}
 \usepackage{mathtools}
 \newcommand{\sen}{\operatorname{sen}\,}
 \newcommand{\senh}{\operatorname{senh}\,}
 \renewcommand{\sin}{\operatorname{sen}\,}
 \renewcommand{\sinh}{\operatorname{senh}\,}
 \begin{document}  \begin{pspicture}(-6,-1.5) (6,3.7)
 \psset{xunit=1,yunit=1}
  \psaxes[labels=y]{->}(0,0)(-3.5,-.5)(3.5,3.5)
  \psline[linecolor=blue,linewidth=2pt]{-}(-3,0)(-3,1.4)
  \psline[linecolor=blue,linewidth=2pt]{-}(-1,0)(-1,3)
	\psline[linecolor=blue,linewidth=2pt]{-}(1,0)(1,3)
	\psline[linecolor=blue,linewidth=2pt]{-}(3,0)(3,1.4)
  \rput(.5,3.4){$|C_n|$}
  \rput(3.5,-.2){$w_n$}
	\rput(-3.2,-.4){$-6\pi$}
  \rput(-2.2,-.4){$-4\pi$}
  \rput(-1.2,-.4){$-2\pi$}
    \rput(1,-.4){$2\pi$}
  \rput(2,-.4){$4\pi$}
   \rput(3,-.4){$6\pi$}
\end{pspicture}
\end{document}