
 \documentclass{standalone}
 \usepackage{pstricks-add}
 \usepackage{pstricks,pst-plot}
 \usepackage[dvips]{graphicx}
 \usepackage{pst-math}
 \usepackage{pst-plot}
 \usepackage{pst-circ}
 \usepackage[brazil]{babel}
 \usepackage[utf8]{inputenc}
 \usepackage[T1]{fontenc}
 \usepackage{amsmath}
 \usepackage{amssymb}
 \usepackage{amsthm}
 \usepackage{mathtools}
 \newcommand{\sen}{\operatorname{sen}\,}
 \newcommand{\senh}{\operatorname{senh}\,}
 \renewcommand{\sin}{\operatorname{sen}\,}
 \renewcommand{\sinh}{\operatorname{senh}\,}
 \begin{document}\psset{xunit =1cm,yunit=2cm, linewidth=1\pslinewidth}
 \begin{pspicture}(-1.3,-1.3)(4.3,1.3)
 \psaxes{->}(0,0)(-1.1,-1.2)(6.2,1.2)
\psplot[plotstyle=curve,linewidth=2\pslinewidth,linecolor=blue]{-1}{0}{0}
\psplot[plotstyle=curve,linewidth=2\pslinewidth,linecolor=blue]{0}{1}{1}
\psplot[plotstyle=curve,linewidth=2\pslinewidth,linecolor=blue]{1}{4}{0}
\psplot[plotstyle=curve,linewidth=2\pslinewidth,linecolor=blue]{4}{5}{1}
\psplot[plotstyle=curve,linewidth=2\pslinewidth,linecolor=blue]{5}{6}{0}
\rput(.3,1.3){$y=f(t)$}
\rput(6.1,.1){$t$}
\end{pspicture}
\end{document}