
 \documentclass{standalone}
 \usepackage{pstricks-add}
 \usepackage{pstricks,pst-plot}
 \usepackage[dvips]{graphicx}
 \usepackage{pst-math}
 \usepackage{pst-plot}
 \usepackage{pst-circ}
 \usepackage[brazil]{babel}
 \usepackage[utf8]{inputenc}
 \usepackage[T1]{fontenc}
 \usepackage{amsmath}
 \usepackage{amssymb}
 \usepackage{amsthm}
 \usepackage{mathtools}
 \newcommand{\sen}{\operatorname{sen}\,}
 \newcommand{\senh}{\operatorname{senh}\,}
 \renewcommand{\sin}{\operatorname{sen}\,}
 \renewcommand{\sinh}{\operatorname{senh}\,}
 \begin{document}\psset{algebraic,unit=1cm,linewidth=1pt}
 \begin{pspicture}(-6.3,-4.0)(6.5,4.0)
 \psaxes[labels=x,ticks=x]{->}(0,0)(-6.0,-3.5)(6.3,3.5)
\psset{linecolor=blue}
\psplot[plotstyle=curve,plotpoints=200]{-1.57}{1.57}{ -2*x }
\psplot[plotstyle=curve,plotpoints=200]{1.57}{4.71}{ -2*x+2*3.14 }
\psplot[plotstyle=curve,plotpoints=200]{4.71}{6}{ -2*x+4*3.14 }
\psplot[plotstyle=curve,plotpoints=200]{-6}{-4.71}{ -2*x-4*3.14 }
\psplot[plotstyle=curve,plotpoints=200]{-4.71}{-1.57}{ -2*x-2*3.14 }
\rput(0,3.7){$\phi(w)$}
\rput(6.4,.3){$w$}
\rput(.4,3.14){$\pi$}
\rput(-.5,-3.14){$-\pi$}
\psline[linecolor=black,linewidth=.4pt](-.14,3.14)(.14,3.14)
\psline[linecolor=black,linewidth=.4pt](-.14,-3.14)(.14,-3.14)
\end{pspicture}
\end{document}