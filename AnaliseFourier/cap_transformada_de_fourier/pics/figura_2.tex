
 \documentclass{standalone}
 \usepackage{pstricks-add}
 \usepackage{pstricks,pst-plot}
 \usepackage[dvips]{graphicx}
 \usepackage{pst-math}
 \usepackage{pst-plot}
 \usepackage{pst-circ}
 \usepackage[brazil]{babel}
 \usepackage[utf8]{inputenc}
 \usepackage[T1]{fontenc}
 \usepackage{amsmath}
 \usepackage{amssymb}
 \usepackage{amsthm}
 \usepackage{mathtools}
 \newcommand{\sen}{\operatorname{sen}\,}
 \newcommand{\senh}{\operatorname{senh}\,}
 \renewcommand{\sin}{\operatorname{sen}\,}
 \renewcommand{\sinh}{\operatorname{senh}\,}
 \begin{document}\psset{unit =1cm, linewidth=1\pslinewidth}
 \begin{pspicture}(-5.3,-.3)(5.5,1.8)
 \psaxes[labels=none]{->}(0,0)(-5.0,-.1)(5.3,1.2)
\psset{linecolor=blue}
\psplot[plotstyle=curve,plotpoints=200]{-5}{-3}{2.718 x 4 add abs -1 mul exp}
\psplot[plotstyle=curve,plotpoints=200]{-3}{-1}{2.718 x 2 add abs -1 mul exp}
\psplot[plotstyle=curve,plotpoints=200]{-1}{1}{2.718 x abs -1 mul exp}
\psplot[plotstyle=curve,plotpoints=200]{1}{3}{2.718 x 2 sub abs -1 mul exp}
\psplot[plotstyle=curve,plotpoints=200]{3}{5}{2.718 x 4 sub abs -1 mul exp}
\rput(0,1.4){$y=f_T(t),~ T=2$}
\rput(5.2,.3){$t$}
\end{pspicture}
\end{document}