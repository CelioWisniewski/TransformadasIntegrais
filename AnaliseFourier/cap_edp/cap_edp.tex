%\documentclass[Main.tex]{subfiles}
%\begin{document}
\chapter{Equa\c{c}\~{o}es diferenciais parciais}

\section{Equação do Calor}
Considere o problema evolutivo de difusão de temperatura numa barra infinita, dado pela equação de calor
\begin{subequations}{\label{eq_calor}}
\begin{eqnarray}
{\label{eq_calor_1}}&&\frac{\partial u}{\partial t}(x,t)=\mu\frac{\partial^2u}{\partial
x^2}(x,t),\quad x\in(-\infty,\infty),\ \ t>0,\\
{\label{eq_calor_2}}&&u(x,0)=f(x)
\end{eqnarray}
\end{subequations}
Tomando a transformada de Fourier desse problema na variável $x$, obtemos
\begin{eqnarray*}
&&\frac{\partial (\mathcal{F}_x\{u(x,t)\})}{\partial t}=-\mu k^2 (\mathcal{F}_x \{u(x,t)\}) \\
&&\mathcal{F}_x\{u(x,0)\}=\mathcal{F}_x \{f(x)\},
\end{eqnarray*}
onde se usou a propriedade \ref{prop_der} da transformada da derivada. Denotando $\mathcal{F}_x\{u(x,t)\}:=U(k,t)$, podemos escrever o problema de uma forma mais limpa:
\begin{eqnarray*}
&&\frac{\partial U}{\partial t}=-\mu k^2 U \\
&&U(k,0)=F(k),
\end{eqnarray*}
Essa é uma equação que pode ser resolvida por vários métodos, entre eles separação de variáveis:
\begin{eqnarray*}
\frac{1}{U}\frac{\partial U}{\partial t}&=&-\mu k^2 \\
&\Downarrow&\\
\ln(U)&=&-\mu k^2t+C \\
&\Downarrow&\\
U&=&e^{-\mu k^2t+C }=Ke^{-\mu k^2t }
\end{eqnarray*}
onde $K=e^C$ é uma constante de integração que é calculada com a condição inicial:
$$
U(k,0)=Ke^{-\mu k^2 \cdot 0 }=F(k)\Rightarrow K=F(k),
$$
Logo,
\begin{equation}\label{eq_trans_eq_calor}
U(k,t) =\mathcal{F}_x\{u(x,t)\}=F(k)e^{-\mu k^2 t}.
\end{equation}
Agora, precisamos calcular a transformada inversa de $U(k,t)$ para obter a solução $u(x,t)$ do problema original. O resultado do exercício \ref{ex_inv_exp_kk} da página \pageref{ex_inv_exp_kk}, temos que
$$
\mathcal{F}^{-1}_k\{e^{-k^2}\}=\frac{1}{2\sqrt{\pi}} e^{-\frac{x^2}{4}}
$$
Usando a propriedade de mudança de escala \ref{prop_mud_esc} com $a=\sqrt{\mu t}$, temos
$$
\mathcal{F}_k^{-1}\{e^{-\mu t k^2}\}=\mathcal{F}_k^{-1}\{e^{-(\sqrt{\mu t} k)^2}\}=\frac{1}{2\sqrt{\pi}}\frac{1}{\sqrt{\mu t}} e^{-\frac{\left(x/\sqrt{\mu t}\right)^2}{4}}
$$
ou seja,
\begin{equation}{\label{eq_calor_02}}
\mathcal{F}_k^{-1}\{e^{-\mu t k^2}\}=\frac{1}{\sqrt{4\pi\mu t}} e^{-\frac{x^2}{4\mu t}}.
\end{equation}
Aplicando esse resultado juntamente com o teorema da convolução descrito na propriedade \ref{prop_teo_conv} na equação (\ref{eq_trans_eq_calor}), obtemos
\begin{equation}{\label{eq_dif_calor}}
u(x,t)=\frac{1}{\sqrt{4\pi \mu t}}\int_{-\infty}^\infty
f(y)e^{-\frac{(x-y)^2}{4\mu t}}dy.
\end{equation}

\begin{ex}Considere o caso particular da equação do calor (\ref{eq_calor}) onde
$$
f(x)=\left\{\begin{array}{ll}
u_0,& |x|\leq 1\\
0,& |x|>1.\\
\end{array}\right.
$$
Então,
\begin{equation*}
u(x,t)=\frac{u_0}{\sqrt{4\pi \mu t}}\int_{-1}^1
e^{-\frac{(x-y)^2}{4\mu t}}dy.
\end{equation*}
Fazendo a mudança de variável $z=\frac{y-x}{2\sqrt{\mu t}}$ e definindo $z_1=\frac{-1-x}{2\sqrt{\mu t}}$ e $z_2=\frac{1-x}{2\sqrt{\mu t}}$, temos:
\begin{equation*}
u(x,t)=\frac{u_02\sqrt{\mu t}}{\sqrt{4\pi \mu t}}\int_{z_1}^{z_2}
e^{-z^2}dz=\frac{u_0}{\sqrt{\pi }}\int_{z_1}^{z_2}
e^{-z^2}dz.
\end{equation*}
Essa expressão pode ser escrita da forma:
\begin{eqnarray*}
u(x,t)&=&\frac{u_0}{\sqrt{\pi }}\int_{z_1}^{0}
e^{-z^2}dz+\frac{u_0}{\sqrt{\pi }}\int_{0}^{z_2}
e^{-z^2}dz\\
&=&\frac{u_0}{\sqrt{\pi }}\int_{0}^{z_2}
e^{-z^2}dz-\frac{u_0}{\sqrt{\pi }}\int_{0}^{z_1}
e^{-z^2}dz\\
&=&\frac{u_0}{2}\hbox{erf}\ \!(z_2)-\frac{u_0}{2}\hbox{erf}\ \!(z_1),
\end{eqnarray*}
onde $\hbox{erf}\ \!(z)$ é a função erro dada por
$$
\hbox{erf}\ \!(x)=\frac{2}{\sqrt{\pi }}\int_{0}^{x}
e^{-z^2}dz
$$
\end{ex}
\begin{ex}{\label{ex_eq_dif_2}}
Considere o fenômeno de difusão de sal ao longo de um cano longo e fino. Suponha que no tempo $t=0$ uma quantidade $Q$ de sal foi introduzida no ponto $x_0$. A equação que modela esse fenômeno é
\begin{eqnarray*}
&&\frac{\partial \rho}{\partial t}=\mu \frac{\partial^2
\rho}{\partial x^2},\quad -\infty<x<\infty,\ \ t>0\\
&&\rho(x,0)=\frac{Q}{A}\delta(x-x_0),
\end{eqnarray*}
onde $A$ é a área da seção transversal do cano e $\rho(x,t)$ é a concentração de sal no ponto $x$ e tempo $t$. A solução desse problema é dada pela equação (\ref{eq_dif_calor}) com $f(x)=\frac{Q}{A}\delta(x-x_0)$:
\begin{equation*}
\rho(x,t)=\frac{Q}{A\sqrt{4\pi \mu t}}\int_{-\infty}^\infty
\delta(y-x_0)e^{-\frac{(x-y)^2}{4\mu t}}dy.
\end{equation*}
Usando a propriedade da filtragem, temos:
\begin{equation*}
\rho(x,t)=\frac{Q}{A\sqrt{4\pi \mu t}}e^{-\frac{(x-x_0)^2}{4\mu t}}.
\end{equation*}


\end{ex}

\section{Equação do calor com termo fonte}
Considere o problema evolutivo de difusão de temperatura numa barra infinita com um termo fonte
\begin{eqnarray*}
&&\frac{\partial u}{\partial t}=\mu \frac{\partial^2u}{\partial
x^2}+f(x,t),\quad x\in(-\infty,\infty),\ \ t>0.\\
&&u(x,0)=0
\end{eqnarray*}
Tomando a transformada de Fourier na variável $x$ obtemos
\begin{eqnarray*}
&&\frac{\partial \mathcal{F}_x\{u(x,t)\}}{\partial t}=-\mu k^2 (\mathcal{F}_x \{u(x,t)\})+\mathcal{F}_x \{f(x,t)\} \\
&&\mathcal{F}_x\{u(x,0)\}=0.
\end{eqnarray*}
Denotando $\mathcal{F}_x\{u(x,t)\}:=U(k,t)$, podemos escrever o problema de uma forma mais limpa:
\begin{eqnarray*}
&&\frac{\partial U}{\partial t}=-\mu k^2 U+F \\
&&U(k,0)=0.
\end{eqnarray*}
Essa equação pode ser resolvida pelo método do fator integrante:
\begin{eqnarray*}
\frac{\partial U}{\partial t}+\mu k^2 U&=&F \\
&\Downarrow&\\
e^{\mu k^2t}\frac{\partial U}{\partial t}+e^{\mu k^2t}\mu k^2 U&=&e^{\mu k^2t}F \\
&\Downarrow&\\
\frac{\partial }{\partial t}\left(Ue^{\mu k^2t}\right)&=&e^{\mu k^2t}F \\
&\Downarrow&\\
 U(k,t)e^{\mu k^2 t}-U(k,0)e^{\mu k^2 \cdot 0}&=&\int_0^t e^{\mu k^2\tau}F(k,\tau) d\tau\\
&\Downarrow&\\
 U(k,t)&=&e^{-\mu k^2 t}\int_0^t e^{\mu k^2\tau}F(k,\tau) d\tau
\end{eqnarray*}
ou seja,
\begin{equation*}
 U(k,t)=\mathcal{F}_x\{u(x,t)\}=\int_0^t
e^{-\mu (t-\tau)k^2}F(k,\tau)d\tau.
\end{equation*}
Agora, precisamos obter a solução do problema original $u(x,t)$, que é a transformada inversa de $U(k,t)$. Usando a equação (\ref{eq_calor_02}), temos que
\begin{equation*}
\mathcal{F}^{-1}_k\{e^{-\mu (t-\tau) k^2}\}=\frac{1}{\sqrt{4\pi\mu (t-\tau)}} e^{-\frac{x^2}{4\mu (t-\tau)}}.
\end{equation*}
Usando o teorema da convolução dado na propriedade \ref{prop_teo_conv}, temos
\begin{eqnarray*}
u(x,t)&=&\int_0^t \left(\mathcal{F}_k^{-1}\left\{
e^{-\mu k^2 (t-\tau)}\right\}\ast f(x,\tau)\right)d\tau\\
&=&\int_0^t \left[\left(\frac{1}{\sqrt{4\pi\mu (t-\tau)}} e^{-\frac{x^2}{4\mu (t-\tau)}}\right) \ast f(x,\tau)\right]d\tau\\
&=&\frac{1}{\sqrt{4\pi\mu}}\int_0^t \left[\frac{1}{\sqrt{ (t-\tau)}} \int_{-\infty}^\infty e^{-\frac{(x-y)^2}{4\mu (t-\tau)}}f(y,\tau)dy\right]d\tau.\\
\end{eqnarray*}

\section{Equação da Onda}
Considere a equação da onda dada por
\begin{eqnarray*}
&&\frac{1}{c^2}\frac{\partial^2 y}{\partial t^2}=\frac{\partial^2
y}{\partial x^2},\quad -\infty<x<\infty,\ \ t>0\\
&&y(x,0)=f(x)\\
&&\frac{\partial y}{\partial t}(x,0)=g(x).
\end{eqnarray*}
Usando a notação
\begin{eqnarray*}
&&Y(k,t)=\mathcal{F} \{y(x,t)\}\\
&&Y(k,0)=\mathcal{F}\{f(x)\}\\
&&\frac{d Y}{d t}(k,0)=\mathcal{F}\{g(x)\}.
\end{eqnarray*}
e tomando a transformada de Fourier da equação, temos
\begin{eqnarray*}
&&\frac{d^2 Y}{d
t^2}(k,t)+c^2k^2Y(k,t)=0\\
&&Y(k,0)=\mathcal{F}\{f(x)\}=F(k)\\
&&\frac{d Y}{d t}(k,0)=\mathcal{F}\{g(x)\}=G(k).
\end{eqnarray*}
A solução desse problema é dada em termos de senos e cossenos:
$$
Y=A\cos(ckt)+B\sen(ckt).
$$
Impondo as condições de contorno, temos:
\begin{eqnarray*}
Y(0)&=&A=F(k)\\
Y'(0)&=&ckB=G(k).
\end{eqnarray*}
ou seja, $A=F(k)$ e $B=\frac{G(k)}{ck}$. Portanto,
\begin{equation*}
Y=F(k)\cos(ckt)+\frac{G(k)}{ck}\sen(ckt).
\end{equation*}
ou
\begin{equation*}
Y=\frac{1}{2}F(k)(e^{ickt}+e^{-ickt})+\frac{G(k)}{2ick}(e^{ickt}-e^{-ickt}).
\end{equation*}

Tomando a transformada de Fourier inversa obtemos
\begin{eqnarray*}
y(x,t)&=&\frac{1}{2}\left(\frac{1}{2
\pi}\int_{-\infty}^\infty
F(k)(e^{ik(x+ct)}+e^{ik(x-ct)})dk\right)+\\&+&\frac{1}{2c}\left(\frac{1}{2
\pi}\int_{-\infty}^\infty
\frac{G(k)}{ik}(e^{ik(x+ct)}-e^{ik(x-ct)})dk \right).
\end{eqnarray*}
Sabemos que
\begin{equation*}
f(x\pm ct)=\frac{1}{2\pi}\int_{-\infty}^\infty
F(k)e^{ik (x\pm ct)}dk,
\end{equation*}
\begin{equation*}
g(x)=\frac{1}{2\pi}\int_{-\infty}^\infty G(k)e^{ik
x}dk
\end{equation*}
e
\begin{equation*}
\int_{x-ct}^{x+ct}
g(\eta)d\eta=\frac{1}{2\pi}\int_{-\infty}^\infty
\frac{G(k)}{i k} (e^{ik(x+ct)}-e^{ik(x-ct)})dk.
\end{equation*}
Portanto,
\begin{equation*}
y(x,t)=\frac{1}{2} \left( f(x+ct)+f(x-ct)
\right)+\frac{1}{2c}\int_{x-ct}^{x+ct}g(\eta)d\eta.
\end{equation*}


\section{Vibrações livres transversais} 
Considere o problema de vibrações livres transversais de uma barra infinita governada por
\begin{eqnarray*}
&&\frac{\partial^4 y}{\partial x^4}+\frac{1}{a^2}\frac{\partial^2
y}{\partial t^2}=0,\quad t>0, x\in (-\infty,\infty)\\
&&y(x,0)=f(x)\\
&&\frac{\partial y}{\partial t}(x,0)=ag''(x).\\
\end{eqnarray*}
Tomando a transformada de Fourier e pondo $Y(k,t)=\mathcal{F}\{y(x,t)\}$,
obtemos
\begin{eqnarray*}
&&\frac{\partial^2 Y}{\partial t^2}+k^4a^2Y=0\\
&&Y(0)=\mathcal{F}\{f\}=F(k)\\
&&\frac{\partial Y}{\partial t}(0)=-ak^2G(k).\\
\end{eqnarray*}
tendo a solução
\begin{equation*}
Y(k,t)=F(k)\cos (ak^2t)-G(k)\sen (ak^2t).
\end{equation*}
Tomando a transformada inversa de Fourier
\begin{equation*}
y(k,t)=\frac{1}{2\pi}\int_{-\infty}^\infty F(k)\cos
(ak^2 t)e^{ik
x}dk-\frac{1}{2\pi}\int_{-\infty}^\infty G(k)\sen
(ak^2 t)e^{ik x}dk.
\end{equation*}
Usando o fato que,
\begin{equation*}
\int_{-\infty}^\infty e^{-k^2 a}e^{ik
x}dk=\frac{\sqrt{\pi}}{\sqrt{a}}e^{-\frac{x^2}{4a}}
\end{equation*}
e
\begin{equation*}
\frac{1}{(a i)^{\frac{1}{2}}}=\frac{1}{\sqrt{a
}}e^{-i\frac{1}{4}\pi},
\end{equation*}
trocamos $a$ por $ai$ para obter
\begin{equation*}
\frac{1}{2\pi}\int_{-\infty}^\infty (\cos (ak^2 t)-i\sen
(ak^2 t))e^{ik
x}dk=\frac{1}{2\sqrt{\pi a}}e^{i\left(\frac{x^2}{4a}-\frac{\pi}{4}\right)}.
\end{equation*}
Tomando as partes real e imaginária nesta equação obtemos que
\begin{equation*}
\frac{1}{2\pi}\int_{-\infty}^\infty \cos (ak^2)e^{ik
x}dk=\frac{\sqrt{2}}{4\sqrt{\pi a}}\left(\cos \left(\frac{x^2}{4a}\right)+\sen\left(
\frac{x^2}{4a}\right)\right)
\end{equation*}
e
\begin{equation*}
\frac{1}{2\pi}\int_{-\infty}^\infty \sen (ak^2)e^{ik
x}dk=\frac{\sqrt{2}}{4\sqrt{\pi a}}\left(\cos\left( \frac{x^2}{4a}\right)-\sen\left(
\frac{x^2}{4a}\right)\right).
\end{equation*}
Utilizando o resultado sobre convoluções dado na propriedade \ref{prop_teo_conv}, obtemos que
\begin{equation*}
\frac{1}{2\pi}\int_{-\infty}^\infty F(k)\cos (ak^2
t)e^{ik x}dk=\frac{\sqrt{2}}{4\sqrt{\pi a t}}\int_{-\infty}^\infty
f(x-y)\left[\cos \left(\frac{y^2}{4at}\right)+\sen\left( \frac{y^2}{4at}\right)\right]dy
\end{equation*}
e 
\begin{equation*}
\frac{1}{2\pi}\int_{-\infty}^\infty G(k)\sen (ak^2
t)e^{ik x}dk=\frac{\sqrt{2}}{4\sqrt{\pi a t}}\int_{-\infty}^\infty
g(x-y)\left[\cos \left(\frac{y^2}{4at}\right)-\sen\left( \frac{y^2}{4at}\right)\right]dy
\end{equation*}
ou seja,
\begin{eqnarray*}
y(x,t)&=&\frac{1}{2\sqrt{2\pi a t}}\int_{-\infty}^\infty
f(x-y)\left(\cos\left( \frac{y^2}{4at}\right)+\sen\left( \frac{y^2}{4at}\right)\right)dy-\\
&-&\frac{1}{2\sqrt{2\pi a t}}\int_{-\infty}^\infty
g(x-y)\left(\cos \left(\frac{y^2}{4at}\right)-\sen\left( \frac{y^2}{4at}\right)\right)dy.
\end{eqnarray*}
Escrevendo $u^2=\dfrac{y^2}{4at}$, obtemos:
\begin{eqnarray*}
y(x,t)&=&\frac{1}{\sqrt{2\pi}}\int_{-\infty}^\infty
f(x-2u a^{\frac{1}{2}}t^{\frac{1}{2}})\left(\cos (u^2)+\sen (u^2)\right)du-\\
&-&\frac{1}{\sqrt{2\pi}}\int_{-\infty}^\infty g(x-2u
a^{\frac{1}{2}}t^{\frac{1}{2}})\left(\sen (u^2)-\cos (u^2)\right)du.
\end{eqnarray*}

\section{Exercícios}

\begin{Exercise}
Um fluido se desloca em um tubo termicamente isolado com velocidade constante $v$ de forma que a evolução da temperatura $u(x,t)$ como uma função da coordenada $x$ e do tempo é descrita pelo seguinte modelo simplificado:
$$u_t-vu_x-u_{xx}=0.$$
Sabendo que no instante $t=0$, a temperatura foi bruscamente aquecida em uma região muito pequena, de forma que podemos considerar
$$u(x,0)=500 \delta(x).$$ 
Use a técnica das transformadas de Fourier para obter a solução desta equação diferencial quando $v=1m/s$.
\end{Exercise}
\begin{Answer}
Aplicamos a transforma de Fourier na variável $x$, obtemos a seguinte expressão para a equação transformada
$$U_t(k,t)-v(ik)U(k,t)-(ik)^2U(k,t)=0$$
onde foi usada a propriedade da derivada.
A condição inicial se torna:
$$U(k,0)=500\int_{-\infty}^\infty \delta(x)e^{-ikx}=500$$
Portanto temos o seguinte problema de valor inicial:
$$U_t(k,t)=(-k^2+ivk)U(k,t)$$
$$U(k,0)=500$$
cuja solução é
$$U(k,t)=500e^{(-k^2+ivk)t}=500 e^{ivkt}e^{-k^2t}$$
A multiplicação por $e^{ivtk}$ indica um deslocamento no eixo $x$. Logo precisamos calcular:
\begin{eqnarray*}
\mathcal{F}^{-1}_x\left\{e^{-k^2t}\right\}&=&\frac{1}{2\pi}\int_{-\infty}^\infty e^{-k^2t} e^{ikx}dk=\frac{1}{\pi}\int_{0}^\infty e^{-k^2t} \cos({ikx})dk\\
&=&\frac{1}{\pi}\frac{\sqrt{\pi}}{2\sqrt{t}}e^{-\frac{x^2}{4t}}=\frac{1}{2\sqrt{\pi t}}e^{-\frac{x^2}{4t}}
\end{eqnarray*}
Portanto
$$u(x,t)=\frac{250}{\sqrt{\pi t}}e^{-\frac{(x+vt)^2}{4t}}=\frac{250}{\sqrt{\pi t}}e^{-\frac{(x+t)^2}{4t}}$$

\end{Answer}

\begin{Exercise} Enconte a solução da equação da onda dada
\begin{eqnarray*}
&&\frac{\partial^2 y}{\partial t^2}=\frac{\partial^2
y}{\partial x^2},\quad -\infty<x<\infty,\ \ t>0\\
&&y(x,0)=f(x)\\
&&\frac{\partial y}{\partial t}(x,0)=g(x).
\end{eqnarray*}
\begin{itemize}
 \item a) Dados $f(x)=e^{-|x|}$ e $g(x)=0$.
  \item b) Dados $f(x)=e^{-3|x|}$ e $g(x)=e^{-x^2}$.
\end{itemize}

\end{Exercise}
\begin{Answer}

\begin{itemize}
 \item a) 
 \begin{equation*}
y(x,t)=\frac{1}{2} \left( e^{-|x+t|}+e^{-|x-t|}
\right).
\end{equation*}
\item b) 
 \begin{equation*}
y(x,t)=\frac{1}{2} \left(  e^{-3|x+t|}+e^{-3|x-t|}
\right)+\frac{1}{2}\int_{x-t}^{x+t}e^{-\eta^2}d\eta.
\end{equation*}
\end{itemize}

\end{Answer}


\begin{Exercise}No exemplo \ref{ex_eq_dif_2}, mostre que a solução satisfaz a condição inicial: 
$$
\lim_{t\to 0}u(x,t)=\frac{Q}{A}\delta(x-x_0)
$$
e, como esperado, vale zero nos limites para infinito:
$$
\lim_{x\to\pm\infty}u(x,t)=0.
$$
\end{Exercise}
\begin{Answer}
\begin{eqnarray*}
\lim_{t\to 0} u(x,t)&=&\frac{Q}{A}\lim_{t\to 0} \frac{1}{\sqrt{4\pi \mu t}}e^{-\frac{(x-x_0)^2}{4\mu t}}\\
&=&\left\{\begin{array}{ll}0,&x\neq x_0\\ \infty, &x=x_0\end{array}\right.
\end{eqnarray*}
Como,
\begin{eqnarray*}
\int_{-\infty}^\infty\frac{1}{\sqrt{4\pi \mu t}}e^{-\frac{(x-x_0)^2}{4\mu t}}dx&=&\frac{2}{\sqrt{4\pi \mu t}}\int_{0}^\infty e^{-\frac{x^2}{4\mu t}}dx\\
&=&\frac{2}{\sqrt{4\pi \mu t}}\frac{\sqrt{\pi} \sqrt{4\mu t}}{2}=1,
\end{eqnarray*}
onde se usou item 8 da tabela de integrais \ref{tab_int_def} com $a=\frac{1}{\sqrt{4\mu t}}$, então
$$
\lim_{t\to 0} u(x,t)=\frac{Q}{A} \delta(x-x_0)
$$
\end{Answer}




\begin{Exercise} Considere uma viga infinita repousada sobre um suporte elástico e $y(x)$ seu deslocamento vertical em cada ponto $x$. Suponha que o suporte exerce uma força de reação proporcional ao deslocamento $y(x)$ e que a viga é carregada em $x=0$ por um força concentrada $P\delta(x)$. A equação que modela o fenômeno é dada por:
$$
EI\frac{d^4 y}{dx^4}=P\delta(x)-Cy(x),\qquad -\infty<x<\infty,
$$
onde $C$ é uma constante de proporcionalidade relacionada ao suporte, $E$ é o módulo de Young e $I$ é o momento de inércia da viga. Calcule o deslocamento $y(x)$ da viga.
\end{Exercise}
\begin{Answer}
\begin{eqnarray*}
y(x)=\frac{P}{EI}\frac{\sqrt{2}}{8a^3}e^{-ax}\sen\left(ax+\frac{\pi}{4}\right),
\end{eqnarray*}
onde $a=\left(\frac{C}{4EI}\right)^{\frac{1}{4}}$.
\end{Answer}


%\end{document}