%Este trabalho está licenciado sob a Licença Creative Commons Atribuição-CompartilhaIgual 3.0 Não Adaptada. Para ver uma cópia desta licença, visite http://creativecommons.org/licenses/by-sa/3.0/ ou envie uma carta para Creative Commons, PO Box 1866, Mountain View, CA 94042, USA.

%%%%%%%%%%%%%%%%%%%%%%%%%%%%%%%%%%%%%%%%%%
% ATENÇÃO
%
% POR SEGURANÇA, EVITE EDITAR ESTE ARQUIVO
%
%%%%%%%%%%%%%%%%%%%%%%%%%%%%%%%%%%%%%%%%%

%%%%%%%%%%%%%%%%%%%%%%%%%%%%%%%%%
%   Predefinicoes
%%%%%%%%%%%%%%%%%%%%%%%%%%%%%%%%%

\newif\ifispdf        % O layout será book?
\newif\ifishtml        % O layout será html?

\def\tfn{config.knd}     % Arquivo que guarda as definições do tipo de saída
\def \tdata{}          % Definições do tipo de saída: book, slide ou html.

\openin1=\tfn\relax    % Leitura das definições de saída
\read1 to \tdata
\closein1

\tdata                 % Definições de saída

%\usepackage{tabu}
%\usepackage{ccicons}
%\usepackage{subfiles}

%opções de linguagem
\usepackage[utf8]{inputenc}
\usepackage[portuges]{babel}
\usepackage[T1]{fontenc}

%Indentar primeiro parágrafo de cada seção
\usepackage{indentfirst} 


\usepackage[hmargin=2.5cm,vmargin=2.5cm]{geometry}
%\usepackage[a5paper]{geometry}
\usepackage{amsfonts}
\usepackage{amsmath}
\usepackage{amsthm}
\usepackage{graphics}
\usepackage[normalem]{ulem}
\usepackage{pstricks,pst-plot}
\usepackage[dvips]{graphicx}%
\usepackage{pst-math}
\usepackage{pst-plot}
\usepackage{pst-circ}

\usepackage[pdfborder={0 0 0 [0 0]},colorlinks=true,linkcolor=blue,citecolor=blue,filecolor=blue,urlcolor=blue]{hyperref}


%\usepackage{hyperref}


%%%%%%%%%%%%%%%%%%%%%%%%%%%%%%%
%%%% Exercises and Answers %%%%
%\usepackage[lastexercise]{exercise}
\usepackage[answerdelayed,lastexercise]{exercise}
\usepackage{chngcntr}
\counterwithin{Exercise}{chapter}
\counterwithin{Answer}{chapter}
\renewcommand{\ExerciseHeaderTitle}{({\it \ExerciseTitle})}
\renewcommand{\ExerciseName}{E}
\renewcommand{\ExerciseHeader}{{\textbf{\large\ExerciseName~\ExerciseHeaderNB\ExerciseHeaderTitle\ExerciseHeaderOrigin}}}
\renewcommand{\ExerciseHeader}{\textbf{\ExerciseName\ \ExerciseHeaderNB.}\,}

% change font for answers header
%\renewcommand{\AnswerHeader}{\tiny\textbf{\ExerciseName\ \ExerciseHeaderNB.}\smallskip}
\renewcommand{\AnswerHeader}{\textbf{\ExerciseName\ \ExerciseHeaderNB.}\smallskip}
% change font for answers list header
\renewcommand{\AnswerListHeader}{{\tiny\textbf{\AnswerListName\
(\ExerciseListName\ \ExerciseHeaderNB)\ ---\ }}}
%%%%%%%%%%%%%%%%%%%%%%%%%%%%%%


\newenvironment{exer}
{\begin{Exercise}}
{\end{Exercise}}

\newenvironment{resp}
{\begin{Answer}\begin{tiny}}
{\end{tiny}\end{Answer}}

\newenvironment{sol}
{\let\oldqedsymbol=\qedsymbol
  \renewcommand{\qedsymbol}{$\Diamond$}
  \begin{proof}[\bfseries\upshape Solução]}
  {\end{proof}
  \renewcommand{\qedsymbol}{\oldqedsymbol}}

%%%%%%%%%%%%%%%%%%%%%%%%%%%%%%
% Exercícios Resolvidos
%%%%%%%%%%%%%%%%%%%%%%%%%%%%%%
\newtheorem{exeresol}{ER}[section]
\newenvironment{resol}
{\let\oldqedsymbol=\qedsymbol
  \renewcommand{\qedsymbol}{$\Diamond$}
  \begin{proof}[\bfseries\upshape Solução]}
  {\end{proof}
  \renewcommand{\qedsymbol}{\oldqedsymbol}}
%%%%%%%%%%%%%%%%%%%%%%%%%%%%%%



%\renewcommand{\ExerciseHeaderTitle}{({\it \ExerciseTitle})}
%\renewcommand{\ExerciseHeader}{{\textbf{\large\ExerciseName~\ExerciseHeaderNB\ExerciseHeaderTitle\ExerciseHeaderOrigin\medskip}}}

%\renewcommand{\AnswerHeader}{\medskip{\textbf{Resposta do exercício \ExerciseHeaderNB}\smallskip}:~}




\newtheorem{teo}{Teorema}
\newtheorem{lem}{Lema}
\newtheorem{prop}{Proposi\c{c}{\~a}o}
\newtheorem{propr}{Propriedade}
\newtheorem{corol}{Corolario}	
\newtheorem{ex}{Exemplo}	
\newtheorem{prob}{Problema}
\newtheorem{obs}{Observa\c{c}{\~a}o}
\newtheorem{defn}{Defini\c{c}{\~a}o}
\newcommand{\senh}{\operatorname{senh}}
\newcommand{\sen}{\operatorname{sen}}


\newcommand{\emconstrucao}{
  \begin{tabular}{|c|}\hline
    Em construção ... Gostaria de participar na escrita deste material? Veja como em:\\
    \url{https://www.ufrgs.br/reamat}\\\hline
  \end{tabular}
}