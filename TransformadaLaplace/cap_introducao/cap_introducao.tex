%Este trabalho está licenciado sob a Licença Creative Commons Atribuição-CompartilhaIgual 3.0 Não Adaptada. Para ver uma cópia desta licença, visite http://creativecommons.org/licenses/by-sa/3.0/ ou envie uma carta para Creative Commons, PO Box 1866, Mountain View, CA 94042, USA.


\chapter{Introdução}
A modelagem de muitos problemas encontrados na física, química e engenharias tais como massa-mola, ou circuitos em série, envolve funç\~{o}es descontínuas. Um exemplo disso é uma função do tipo chave liga/desliga, que é zero o início do fenômeno, depois sobe instantaneamente a um valor constante durante algum tempo e, finalmente, zero novamente. Métodos analíticos para resolver equações diferenciais, como fator integrante, separação de variáveis, coeficientes a determinar e variação de parâmetros, funcionam bem quando as funções e envolvidas são contínuas. O método que vamos introduzir aqui, chamado de transformada de Laplace, resolve esse tipo de problema. Essencialmente, a transformada de Laplace é uma transformação similar a derivação ou integração, pois leva função em outra função. Alem disso, essa transformação leva a derivada de uma função em produtos da função original. Isso significa que essa transformação leva uma equação diferencial a uma nova equação em termos da funç\~{a}o transformada que é algébrica e pode ser resolvida facilmente. Uma vez que a transformada de Laplace é conhecida, temos que calcular a transformada inversa para obter a solução do problema (\cite{ZILL} and \cite{STRAUCH}). 

Para cursar essas disciplina o estudante deve conhecer técnicas básicas estudadas em disciplinas como Cálculo Diferencial e Integral e Equações Diferenciais. Nesse sentido, propomos alguns exercícios revisando algumas técnicas úteis.
\section*{Exercícios}
\begin{exer}
Use as expressões $\cosh x=\frac{e^x+e^{-x}}{2}$, $\senh x=\frac{e^x-e^{-x}}{2}$, $\cos x=\frac{e^{ix}+e^{-ix}}{2}$ e $\sen x=\frac{e^{ix}-e^{-ix}}{2i}$  para calcular as seguintes integrais:
\begin{itemize}
\item[a)] $\int_{0}^{\infty} \sen(t)e^{-t}dt$
\item[b)] $\int_{0}^{\infty} \cos(wt)e^{-t}dt$
\item[c)] $\int_{0}^{\infty} \sen^2(wt)e^{-t}dt$
\item[d)] $\int_{0}^{\infty} \cos^2(wt)e^{-t}dt$
\item[e)] $\int_{0}^{\infty} \senh(t)e^{-2t}dt$
\item[f)] $\int_{0}^{\infty} \cosh(t)e^{-2t}dt$
\end{itemize}
 
\end{exer}
\begin{resp}
\begin{itemize}
\item[a)] $\frac{1}{2}$
\item[b)] $\frac{1}{w^2+1}$
\item[c)] $\frac{2w^2}{4w^2+1}$
\item[d)] $\frac{2w^2+1}{4w^2+1}$
\item[e)] $\frac{1}{3}$
\item[f)] $\frac{2}{3}$
\end{itemize}
 
\end{resp}



\begin{exer}
 Calcule o valor da integral
\begin{equation}\int_0^\infty \sen(wt) e^{-st}dt 
\end{equation}

como uma função de $w$ e $s$ sabendo que $s$ e $w$ são constantes reais positivas.
\end{exer}
\begin{resp}
 $\frac{w}{s^2+w^2}$
\end{resp}


\begin{exer} Calcule o valor da integral
\begin{equation}\int_0^a  e^{-st}dt
 \end{equation}

como uma função de $a$ e $s$ sabendo que $a$ e $s$ são constantes reais positivas.
\end{exer}
\begin{resp}
 $\frac{1-e^{-as}}{s}$
\end{resp}


\begin{exer}  Mostre que se $|x|<1$ então
\begin{itemize}
\item[a)] $\sum_{k=0}^\infty x^k=\frac{1}{1-x}$
\item[b)] $\sum_{k=0}^\infty kx^k=\frac{x}{(1-x)^2}$
\end{itemize}
\end{exer}

\begin{exer} Use o resultado anterior para resolver os seguintes somatórios
\begin{itemize}
\item[a)]$\sum_{k=0}^\infty e^{-sk}$
\item[b)]$\sum_{k=0}^\infty (-1)^ke^{-sk}$
\item[c)]$\sum_{k=0}^\infty ke^{-sk}$
\end{itemize}
onde $s$ é uma constante real positiva. 
\end{exer}
\begin{resp}
\begin{itemize}
  \item[a)]$\frac{1}{1-e^{-s}}$
  \item[b)]$\frac{1}{1+e^{-s}}$
  \item[c)]$\frac{e^{-s}}{\left(1-e^{-s}\right)^2}$
\end{itemize}
\end{resp}



\begin{exer} Use a técnica de integração por partes para realizar as seguintes integrais:
\begin{itemize}
\item[a)]$\int_0^\infty t e^{-t}dt$
\item[b)]$\int_0^\infty t^2 e^{-t} dt$
\end{itemize}
\end{exer}

\begin{resp}
\begin{itemize}
  \item[a)]$1$
  \item[b)]$2$
  \end{itemize}
\end{resp}


\begin{exer} Calcule a integral
\begin{equation}I=\int_0^\infty \left|\sen(\pi t)\right| e^{-t}dt
 \end{equation}
escrevendo-o como o somatório
\begin{equation}I=\sum_{k=0}^\infty(-1)^k\int_{k}^{k+1} \sen(\pi t) e^{-t}dt.
\end{equation}
\end{exer}

\begin{resp}
\begin{eqnarray*}
I&=&\sum_{k=0}^\infty(-1)^k\int_{k}^{k+1} \sen(\pi t) e^{-t}dt\\
&=&\frac{\pi}{s^2+\pi^2}\sum_{k=0}^\infty\left[e^{-ks}+e^{-(k+1)s}\right]\\
&=&\frac{\pi}{s^2+\pi^2}\frac{1+e^{-s}}{1-e^{-s}}\\
&=&\frac{\pi}{s^2+\pi^2}\frac{e^{s/2}+e^{-s/2}}{e^{s/2}-e^{-s/2}}\\
&=&\frac{\pi}{s^2+\pi^2}\coth(s/2)\\
\end{eqnarray*}

\end{resp}


%\end{document}